\documentclass[a4paper,12pt,twoside]{memoir}

% Castellano
\usepackage[spanish,es-tabla]{babel}
\selectlanguage{spanish}
\usepackage[utf8]{inputenc}
\usepackage[T1]{fontenc}
\usepackage{lmodern} % Scalable font
\usepackage{microtype}
\usepackage{placeins}

\RequirePackage{booktabs}
\RequirePackage[table]{xcolor}
\RequirePackage{xtab}
\RequirePackage{multirow}

% Links
\PassOptionsToPackage{hyphens}{url}\usepackage[colorlinks]{hyperref}
\hypersetup{
	allcolors = {red}
}

% Ecuaciones
\usepackage{amsmath}

% Rutas de fichero / paquete
\newcommand{\ruta}[1]{{\sffamily #1}}

% Párrafos
\nonzeroparskip

% Huérfanas y viudas
\widowpenalty100000
\clubpenalty100000

% Imagenes
\usepackage{graphicx}
\newcommand{\imagen}[3]{
	\begin{figure}[!ht]
		\centering
		\includegraphics[width=#3\textwidth]{#1}
		\caption{#2}\label{fig:#1}
	\end{figure}
	\FloatBarrier
}

\newcommand{\imagenflotante}[2]{
	\begin{figure}%[!ht]
		\centering
		\includegraphics[width=0.9\textwidth]{#1}
		\caption{#2}\label{fig:#1}
	\end{figure}
}



% El comando \figura nos permite insertar figuras comodamente, y utilizando
% siempre el mismo formato. Los parametros son:
% 1 -> Porcentaje del ancho de página que ocupará la figura (de 0 a 1)
% 2 --> Fichero de la imagen
% 3 --> Texto a pie de imagen
% 4 --> Etiqueta (label) para referencias
% 5 --> Opciones que queramos pasarle al \includegraphics
% 6 --> Opciones de posicionamiento a pasarle a \begin{figure}
\newcommand{\figuraConPosicion}[6]{%
  \setlength{\anchoFloat}{#1\textwidth}%
  \addtolength{\anchoFloat}{-4\fboxsep}%
  \setlength{\anchoFigura}{\anchoFloat}%
  \begin{figure}[#6]
    \begin{center}%
      \Ovalbox{%
        \begin{minipage}{\anchoFloat}%
          \begin{center}%
            \includegraphics[width=\anchoFigura,#5]{#2}%
            \caption{#3}%
            \label{#4}%
          \end{center}%
        \end{minipage}
      }%
    \end{center}%
  \end{figure}%
}

%
% Comando para incluir imágenes en formato apaisado (sin marco).
\newcommand{\figuraApaisadaSinMarco}[5]{%
  \begin{figure}%
    \begin{center}%
    \includegraphics[angle=90,height=#1\textheight,#5]{#2}%
    \caption{#3}%
    \label{#4}%
    \end{center}%
  \end{figure}%
}
% Para las tablas
\newcommand{\otoprule}{\midrule [\heavyrulewidth]}
%
% Nuevo comando para tablas pequeñas (menos de una página).
\newcommand{\tablaSmall}[5]{%
 \begin{table}[!ht]
  \begin{center}
   \rowcolors {2}{gray!35}{}
   \begin{tabular}{#2}
    \toprule
    #4
    \otoprule
    #5
    \bottomrule
   \end{tabular}
   \caption{#1}
   \label{tabla:#3}
  \end{center}
 \end{table}
}

%
% Nuevo comando para tablas pequeñas (menos de una página).
\newcommand{\tablaSmallSinColores}[5]{%
 \begin{table}[H]
  \begin{center}
   \begin{tabular}{#2}
    \toprule
    #4
    \otoprule
    #5
    \bottomrule
   \end{tabular}
   \caption{#1}
   \label{tabla:#3}
  \end{center}
 \end{table}
}

\newcommand{\tablaApaisadaSmall}[5]{%
\begin{landscape}
  \begin{table}
   \begin{center}
    \rowcolors {2}{gray!35}{}
    \begin{tabular}{#2}
     \toprule
     #4
     \otoprule
     #5
     \bottomrule
    \end{tabular}
    \caption{#1}
    \label{tabla:#3}
   \end{center}
  \end{table}
\end{landscape}
}

%
% Nuevo comando para tablas grandes con cabecera y filas alternas coloreadas en gris.
\newcommand{\tabla}[6]{%
  \begin{center}
    \tablefirsthead{
      \toprule
      #5
      \otoprule
    }
    \tablehead{
      \multicolumn{#3}{l}{\small\sl continúa desde la página anterior}\\
      \toprule
      #5
      \otoprule
    }
    \tabletail{
      \hline
      \multicolumn{#3}{r}{\small\sl continúa en la página siguiente}\\
    }
    \tablelasttail{
      \hline
    }
    \bottomcaption{#1}
    \rowcolors {2}{gray!35}{}
    \begin{xtabular}{#2}
      #6
      \bottomrule
    \end{xtabular}
    \label{tabla:#4}
  \end{center}
}

%
% Nuevo comando para tablas grandes con cabecera.
\newcommand{\tablaSinColores}[6]{%
  \begin{center}
    \tablefirsthead{
      \toprule
      #5
      \otoprule
    }
    \tablehead{
      \multicolumn{#3}{l}{\small\sl continúa desde la página anterior}\\
      \toprule
      #5
      \otoprule
    }
    \tabletail{
      \hline
      \multicolumn{#3}{r}{\small\sl continúa en la página siguiente}\\
    }
    \tablelasttail{
      \hline
    }
    \bottomcaption{#1}
    \begin{xtabular}{#2}
      #6
      \bottomrule
    \end{xtabular}
    \label{tabla:#4}
  \end{center}
}

%
% Nuevo comando para tablas grandes sin cabecera.
\newcommand{\tablaSinCabecera}[5]{%
  \begin{center}
    \tablefirsthead{
      \toprule
    }
    \tablehead{
      \multicolumn{#3}{l}{\small\sl continúa desde la página anterior}\\
      \hline
    }
    \tabletail{
      \hline
      \multicolumn{#3}{r}{\small\sl continúa en la página siguiente}\\
    }
    \tablelasttail{
      \hline
    }
    \bottomcaption{#1}
  \begin{xtabular}{#2}
    #5
   \bottomrule
  \end{xtabular}
  \label{tabla:#4}
  \end{center}
}



\definecolor{cgoLight}{HTML}{EEEEEE}
\definecolor{cgoExtralight}{HTML}{FFFFFF}

%
% Nuevo comando para tablas grandes sin cabecera.
\newcommand{\tablaSinCabeceraConBandas}[5]{%
  \begin{center}
    \tablefirsthead{
      \toprule
    }
    \tablehead{
      \multicolumn{#3}{l}{\small\sl continúa desde la página anterior}\\
      \hline
    }
    \tabletail{
      \hline
      \multicolumn{#3}{r}{\small\sl continúa en la página siguiente}\\
    }
    \tablelasttail{
      \hline
    }
    \bottomcaption{#1}
    \rowcolors[]{1}{cgoExtralight}{cgoLight}

  \begin{xtabular}{#2}
    #5
   \bottomrule
  \end{xtabular}
  \label{tabla:#4}
  \end{center}
}



\graphicspath{ {./img/} }

% Capítulos
\chapterstyle{bianchi}
\newcommand{\capitulo}[2]{
	\setcounter{chapter}{#1}
	\setcounter{section}{0}
	\setcounter{figure}{0}
	\setcounter{table}{0}
	\chapter*{#2}
	\addcontentsline{toc}{chapter}{#2}
	\markboth{#2}{#2}
}

% Apéndices
\renewcommand{\appendixname}{Apéndice}
\renewcommand*\cftappendixname{\appendixname}

\newcommand{\apendice}[1]{
	%\renewcommand{\thechapter}{A}
	\chapter{#1}
}

\renewcommand*\cftappendixname{\appendixname\ }

% Formato de portada
\makeatletter
\usepackage{xcolor}
\newcommand{\tutor}[1]{\def\@tutor{#1}}
\newcommand{\course}[1]{\def\@course{#1}}
\definecolor{cpardoBox}{HTML}{E6E6FF}
\def\maketitle{
  \null
  \thispagestyle{empty}
  % Cabecera ----------------
\noindent\includegraphics[width=\textwidth]{cabecera}\vspace{1cm}%
  \vfill
  % Título proyecto y escudo informática ----------------
  \colorbox{cpardoBox}{%
    \begin{minipage}{.8\textwidth}
      \vspace{.5cm}\Large
      \begin{center}
      \textbf{TFG del Grado en Ingeniería Informática}\vspace{.6cm}\\
      \textbf{\LARGE\@title{}}
      \end{center}
      \vspace{.2cm}
    \end{minipage}

  }%
  \hfill\begin{minipage}{.20\textwidth}
    \includegraphics[width=\textwidth]{escudoInfor}
  \end{minipage}
  \vfill
  % Datos de alumno, curso y tutores ------------------
  \begin{center}%
  {%
    \noindent\LARGE
    Presentado por \@author{}\\ 
    en Universidad de Burgos --- \@date{}\\
    Tutores: \@tutor{}\\
  }%
  \end{center}%
  \null
  \cleardoublepage
  }
\makeatother

\newcommand{\nombre}{Diego Garrido Calvo} %%% cambio de comando

% Datos de portada
\title{Control de un molino eólico utilizando aprendizaje por refuerzo profundo}
\author{\nombre}
\tutor{Daniel Sarabia Ortiz y Roberto Carlos Casado Vara}
\date{\today}

\begin{document}

\maketitle


\newpage\null\thispagestyle{empty}\newpage


%%%%%%%%%%%%%%%%%%%%%%%%%%%%%%%%%%%%%%%%%%%%%%%%%%%%%%%%%%%%%%%%%%%%%%%%%%%%%%%%%%%%%%%%
\thispagestyle{empty}


\noindent\includegraphics[width=\textwidth]{cabecera}\vspace{1cm}

\noindent D. Daniel Sarabia Ortiz, profesor del departamento de Ingeniería electromecánica, área de Ingeniería de Sistemas y Automática.

\noindent Expone:

\noindent Que el alumno D. \nombre, con DNI 71364562R, ha realizado el Trabajo final de Grado en Ingeniería Informática titulado Control de un molino eólico utilizando aprendizaje por refuerzo profundo. 

\noindent Y que dicho trabajo ha sido realizado por el alumno bajo la dirección de los que suscriben, en virtud de lo cual se autoriza su presentación y defensa.

\begin{center} %\large
En Burgos, {\large \today}
\end{center}

\vfill\vfill\vfill

% Author and supervisor
\begin{minipage}{0.45\textwidth}
\begin{flushleft} %\large
Vº. Bº. del Tutor:\\[2cm]
D. Daniel Sarabia Ortiz
\end{flushleft}
\end{minipage}
\hfill
\begin{minipage}{0.45\textwidth}
\begin{flushleft} %\large
Vº. Bº. del co-tutor:\\[2cm]
D. Roberto Carlos Casado Vara
\end{flushleft}
\end{minipage}
\hfill

\vfill







\newpage\null\thispagestyle{empty}\newpage




\frontmatter

% Abstract en castellano
\renewcommand*\abstractname{Resumen}
\begin{abstract}
El objetivo de este proyecto es el desarrollo de un algoritmo de aprendizaje por refuerzo con la finalidad de simular el control de un molino eólico. 

El algoritmo deberá recibir por parámetro tanto el viento como la energía a producir por el molino y este deberá calcular el ángulo que deben de tomar las palas para lograr producir la energía indicada, teniendo en cuenta las perturbaciones como los cambios en la fuerza del viento.
\end{abstract}

\renewcommand*\abstractname{Descriptores}
\begin{abstract}
Aprendizaje por refuerzo, molinos eólicos, Anaconda, Jupyter Notebook, Python, Keras  \ldots
\end{abstract}

\clearpage

% Abstract en inglés
\renewcommand*\abstractname{Abstract}
\begin{abstract}
The goal of this project is the development of a reinforcement learning algorithm in order to simulate control of an Eolic windmill.

The algorithm should receive as parameter both the wind, and the energy we want the windmill to produce, and it must calculate the angle that the blades should take to achieve to produce the energy we indicated, all these considering disturbances as changes into the wind force.
\end{abstract}

\renewcommand*\abstractname{Keywords}
\begin{abstract}
Reinforcement learning, eolic windmills, Anaconda, Jupyter Notebook, Python, Keras
\end{abstract}

\clearpage

% Indices
\tableofcontents

\clearpage

\listoffigures

\clearpage

\listoftables
\clearpage

\mainmatter
\capitulo{1}{Introducción}

El interés por el uso y el funcionamiento de las energías renovables ha crecido mucho en los últimos años y, en particular, la energía producida por el viento.
La mejora en la eficiencia de las plantas de producción eólica sigue siendo un desafío pero también una necesidad absoluta.

Al mismo tiempo, se han realizado un gran número de avances en cuanto a inteligencia artificial\cite{wiki:IA}, la cual vemos cada vez más implementada en distintos aparatos que usamos en la vida cotidiana. Estos avances han llevado a la aparición del aprendizaje por refuerzo profundo\cite{RL}, el cual consiste en el aprendizaje por parte de un agente computacional a tomar decisiones por un método de prueba y error, pretendiendo alcanzar una solución óptima a un problema.

La posibilidad de combinar esto con las ecuaciones de una turbina eólica, sumado al problema de los costes que conlleva el tener un modelo físico de la propia turbina para realizar pruebas, han llevado a la idea de este proyecto, en el cual se modeliza el funcionamiento de un molino eólico a través de distintas herramientas de software.


\section{Estructura de la memoria}

\begin{itemize}
    \item \textbf{Introducción}: breve descripción del proyecto y del problema a resolver, junto a la estructura de la memoria y los materiales adjuntos.
    \item \textbf{Objetivos del proyecto}: objetivos generales, técnicos y personales que se pretenden lograr con la realización del proyecto.
    \item \textbf{Conceptos teóricos}: Breve descripción de una serie de conceptos que son necesarios conocer para la correcta comprensión del proyecto.
    \item \textbf{Técnicas y herramientas}: aquí se describen la serie de elementos que se han utilizado durante todo el proyecto, así como distintas técnicas que han permitido facilitar el desarrollo del mismo.
    \item \textbf{Aspectos relevantes del desarrollo del proyecto}: descripción general de la realización del proyecto.
    \item \textbf{Trabajos relacionados}: trabajos parecidos o similares al desarrollado.
    \item \textbf{Conclusiones y líneas de trabajo futuras}: pensamientos y opinión personal tras la realización del proyecto, así como mejoras y posibilidades de seguir desarrollando y mejorando el proyecto por otras ramas.
\end{itemize}

\section{Materiales adjuntos}

\begin{itemize}
    \item Link al repositorio de GitHub con el código: \href{https://github.com/Di3go99/GII-21.35-Control-de-un-molino-eolico-usando-Aprendizaje-por-refuerzo-profundo}{https://github.com/Di3go99/GII-21.35-Control-de-un-molino-eolico-usando-Aprendizaje-por-refuerzo-profundo}.
    \item Memoria detallada del proyecto.
    \item Anexos.
\end{itemize}
\capitulo{2}{Objetivos del proyecto}

En este apartado se indican los distintos objetivos que han motivado al alumno a la realización del proyecto.

\section{Objetivos generales}

\begin{itemize}
    \item Modelar el comportamiento de una turbina eólica a partir de un algoritmo de aprendizaje por refuerzo, el cuál sea capaz de ajustarse automáticamente a una demanda de energía dada.
    \item Calcular y aplicar las ecuaciones que intervienen en el funcionamiento de una turbina eólica.
    \item Diseñar una interfaz-web sencilla para interactuar con el software del modelo.
    \item Facilitar el entendimiento de la respuesta del molino mediante gráficas.
\end{itemize}


\section{Objetivos técnicos}

\begin{itemize}
    \item Desarrollar una aplicación el Python\cite{Python} mediante la plataforma Jupyter Notebook\cite{Jupyter}.
    \item Utilizar los paquetes de \textit{Keras-rl} y \textit{Tensorflow} para el aprendizaje por refuerzo.
    \item Utilizar los paquetes de \textit{Numpy} y \textit{MatbPlot} para la representación gráfica de resultados matemáticos.
    \item Utilizar Flask\cite{Flask} y HTML\cite{HTML} para el desarrollo de una interfaz web sencilla.
    \item Utilizar una estructura de red neuronal DQN\cite{wiki:DQN} para entrenar un agente.
    \item Utilizar GitHub\cite{GitHub} como repositorio para ir publicando las distintas versiones del proyecto y tener un registro de su evolución.
    \item Aplicar la metodología Scrum\cite{MetScrum} para organizar las reuniones y dividir las tareas en \textit{Sprints}.
    \item Utilizar la aplicación web Microsoft Sharepoint\cite{Sharepoint} para compartir artículos, foros, libros y cualquier tipo de documentación que pueda ser de ayuda para la realización del proyecto.
    \item Documentar toda la información del proyecto utilizando \LaTeX\cite{wiki:latex}.
\end{itemize}


\section{Objetivos personales}

\begin{itemize}
    \item Comprender el funcionamiento del aprendizaje por refuerzo, así como entender la importancia del mismo y saber aplicarlo a un modelo software.
    \item Profundizar en el campo de la inteligencia artificial.
    \item Aplicar el mayor número de conocimientos obtenidos durante la carrera.
    \item Aprender a utilizar todas las aplicaciones mencionadas en las \textit{Técnicas y herramientas} y afianzar conocimientos en aquellas usadas previamente.
\end{itemize}
\capitulo{3}{Conceptos teóricos}

A continuación, se van a exponer una serie de conceptos teóricos para dar al usuario una base de conocimiento para entender el proyecto.


\section{Aprendizaje por refuerzo}

\subsection{¿Qué es el aprendizaje por refuerzo?}

El aprendizaje por refuerzo es un área del \textit{machine learning} (aprendizaje automático) que se basa en hacer aprender a una inteligencia artificial, a través de un sistema de recompensas, cuál es la acción más óptima a tomar, en un entorno en el cual influyen múltiples variables que cambian a lo largo del tiempo, para lograr un objetivo preestablecido.

En algunas situaciones, la acción tomada puede afectar incluso a la siguiente situación y, por lo tanto, afectar al resto de recompensas. Tanto la prueba y error, como la dependencia de la acción sobre la recompensa, son las principales características del aprendizaje por refuerzo.

\subsection{Fases del aprendizaje por refuerzo}

Las fases del aprendizaje por refuerzo se pueden dividir de la siguiente manera:

\begin{itemize}
    \item Observar: se realiza una observación del entorno.
    \item Decidir: se decide la acción a realizar.
    \item Actuar: se ejecuta la acción elegida, lo que provoca cambios en el entorno.
    \item Recompensa: se recibe una recompensa según los resultados de la acción tomada.
    \item Aprender: se observa la recompensa obtenida y se aprende de los resultados.
    \item Repetición: se repite el proceso las veces que sean necesarias para alcanzar una estrategia óptima.
\end{itemize}

\subsection{Elementos del aprendizaje por refuerzo}

Antes de entrar en mayor detalle sobre cómo se ha aplicado esta técnica en el proyecto, se procede a definir los distintos elementos que componen un modelo de aprendizaje por refuerzo:
\begin{itemize}
    \item Agente: el agente es el "aprendiz" del que se ha hablado anteriormente. Es quien se encarga de tomar las acciones y observar el resultado tras aplicarlas.
    \item Entorno: el entorno es el espacio de pruebas con el cual el agente va a interactuar para obtener una respuesta. Aquí es donde se declaran el número de acciones, se actualiza el estado y se construye la función de recompensa.
    \item Acciones: son las distintas decisiones que puede tomar el agente.
    \item Estado: es la respuesta que el agente recibe del entorno.
    \item Recompensa: es un valor numérico que recibe el agente dependiendo del resultado de la acción elegida en ese instante. Las acciones con mayor recompensa indican que están mas cerca de alcanzar el objetivo final.
    \item Política: es la estrategia que define cómo se va a comportar el agente para alcanzar el objetivo. La política toma las posibles acciones del agente y elige qué acción aplicar sobre el entorno haciendo uso de un algoritmo predefinido.
    \item Función de valor: es la recompensa total que se puede esperar de un estado. Mientras que la recompensa nos indica lo buena que es una acción inmediata, la función indica qué es bueno a largo plazo.
    \item Modelo: el modelo nos permite predecir cómo se va a comportar el entorno cuando ejecutemos acciones sobre éste. 
    El modelo no siempre es posible tenerlo. Si lo tenemos es más fácil para el agente tomar acciones ya que se puede preguntar al modelo cuál es la acción óptima en cada instante. Si no lo tenemos, el agente deberá probar todas las acciones posibles sobre el entorno en cada estado para aprender cuál es la forma óptima de llegar al objetivo.
\end{itemize}

\subsection{Aplicación del aprendizaje por refuerzo al proyecto}

En este proyecto se hace uso del aprendizaje por refuerzo para modelar un agente, el cual va a ser capaz de alcanzar una potencia de referencia indicada, partiendo de un punto inicial aleatorio, aumentando o disminuyendo el ángulo de las aspas del molino.

Para ello, se ha optado por el algoritmo DQN, mediante el cual se ha modelado una estructura de red neuronal, la cual va a utilizar el agente para entrenarse sobre el entorno construido.

Se parte de un total de 3 acciones; aumentar, disminuir y mantener el ángulo de las aspas del molino. A continuación, el agente toma una acción y se actualiza el estado según la acción tomada. Tras esto se otorga una recompensa según si se ha acercado, mantenido o alejado del objetivo.
Por último el entorno devuelve el estado actualizado y la recompensa obtenida, la cual es observada por el agente para decidir cuál será la siguiente acción a tomar.


\section{Turbina eólica}

Una turbina eólica es un dispositivo mecánico que convierte la energía eólica en energía eléctrica.

El problema de control que plantea este proyecto, es el de alcanzar la potencia generada óptima para un ángulo de las aspas predeterminado.

\label{fig:turbina}
\imagen{Turbina.png}{Mecanismo de rotación y generador eléctrico.}{0.9}

En la imagen \ref{fig:turbina}, podemos ver las distintas variables que influyen en el funcionamiento tanto de la turbina, cómo del generador eléctrico.

Según el artículo \cite{control_multivariable}, las ecuaciones para el modelo estático de la turbina son las siguientes:

\begin{equation}
    \tau_{a} = \frac{1}{2} \rho \pi R^2 v^3 C_{q}(\lambda,\beta)
    \label{eq:torque}
\end{equation}

\begin{equation}
    P_{g} = \frac{1}{2} \rho \pi R^2 v^3 C_{p}(\lambda,\beta)
    \label{eq:potenciaGenerada}
\end{equation}

\begin{equation}
    \tau_{a} = \frac{P_{g}}{\omega_{r}}
    \label{eq:relacionPotenciaTorque}
\end{equation}

\begin{equation}
    \lambda = \frac{R \omega_{r}}{v}
    \label{eq:tipSpeedRatio}
\end{equation}

En la figura \ref{eq:torque}, podemos ver la fórmula del \textit{torque}, el cual es el par generado por la fuerza del viento.
Mediante la fórmula de la figura \ref{eq:potenciaGenerada} calculamos la cantidad de potencia generada por la turbina.
Tras esto, tenemos la relación entre el torque y la potencia generada en la figura \ref{eq:relacionPotenciaTorque}.
El \textit{Ratio de velocidad punta} de la figura \ref{eq:tipSpeedRatio} es la relación entre la velocidad angular y la velocidad del viento.

Para entender mejor estas ecuaciones, se pasa a definir cada una de las variables:

\begin{itemize}
    \item \textit{v}: viento en metros por segundo (\textit{m/s}).
    \item \textit{$\lambda$}: Ratio de velocidad punta.
    \item \textit{$\beta$}: ángulo de \textit{pitch} en grados.
    \item \textit{$\tau$a}: par generado por la fuerza del viento.
    \item \textit{$\rho$}: densidad del viento (kg/\textit{m3}).
    \item \textit{$\omega$r}: velocidad angular de la flecha en revoluciones por minuto (\textit{rpm}).
    \item \textit{R}: radio de las aspas en metros (\textit{m}).
    \item \textit{Cq}: coeficiente de eficiencia en el par.
    \item \textit{Cp}: coeficiente de eficiencia en la potencia generada.
    \item \textit{Pg}: potencia generada en kilowatios (\textit{kW}).
\end{itemize}

En la tabla \ref{tabla:constantes} se muestra el valor de las constantes definidas.

\tablaSmall{Constantes modelo turbina}{l c}{constantes}
{ \multicolumn{1}{l}{Constante} & Valor\\}{ 
\textit{v} & 10 m/s\\
\textit{$\rho$} & 1,225 kg/m^3\\
\textit{R} & 2 m\\
}

Para reducir la complejidad de la matemática y evitar utilizar excesivo tiempo en esta parte del proyecto, se ha simplificado el modelo que representa la producción de potencia en una ecuación diferencial:

\begin{equation}
    \alpha \frac{dP_{g}(t)}{dt} = \frac{1}{2} \rho \pi R^2 v^3 C_{p}(\lambda,\beta) - P_{g}(t)
    \label{eq:modeloReducido}
\end{equation}

En \ref{eq:modeloReducido} se puede ver la fórmula final que será utilizada para calcular la potencia del modelo.

Según este modelo podemos aumentar o disminuir la potencia generada según el ángulo de pitch a través de la función \textit{Cp} (eficiencia de la potencia generada). Para ello se ha optado por linealizar la función, tomando un valor fijo de la razón \textit{$\lambda$} de 6, que es el valor que permite el máximo coeficiente \textit{Cp}, lo que hace que el ángulo de las aspas pueda oscilar entre 5º y 14º (puede verse en más detalle en \ref{fig:GraficaEficiencia}). Esto se ha hecho teniendo en cuenta el punto de funcionamiento más óptimo del molino y para no complicar mucho el modelo, ya que habría que depender de dos variables de entrada.
De esta forma, la fórmula quedaría como se ve en \ref{eq:coeficientePotencia}.

\imagen{GraficaEficiencia.png}{Comportamiento de la eficiencia en función del ángulo \textit{$\beta$} y de la razón \textit{$\lambda$}.}{0.8}
\label{fig:GraficaEficiencia}

\begin{equation}
    C_{p}(\lambda,\beta) = -0.0422 \beta + 0.5911
    \label{eq:coeficientePotencia}
\end{equation}

\section{Método de Euler}

El método de Euler es un procedimiento matemático de integración numérica para resolver ecuaciones diferenciales. Este método se usa para calcular la traza de una curva conociendo su punto de comienzo.
Explicado de manera informal el método consiste en, a partir de la ecuación diferencial, calcular la pendiente de la curva en dicho punto y con ello la recta tangente a la curva. Tras esto, damos un pequeño paso sobre la recta para tomar un nuevo punto y repetimos lo mismo realizado anteriormente.
Tras varios pasos, habremos formado una curva a través de esos puntos. Es cierto que siempre va a existir un error entre la curva calculada y la original, aunque este error puede ser minimizado disminuyendo el tamaño de los pasos al avanzar sobre la recta tangente.

Se procede a detallar cómo se ha resuelto el modelo reducido de la turbina utilizando este método:

    1. Primero, se aproxima la derivada de la función por un cociente de incrementos. Quedaría lo que se muestra en la figura \ref{eq:euler1}.
    
\begin{equation}
\begin{split}
    \alpha \frac{dP_{g}(t)}{dt} = \frac{1}{2} \rho \pi R^2 v^3 C_{p}(\lambda,\beta) - P_{g}(t) \Rightarrow P_{g}(t + \Delta t) = \\
    \frac{1}{2 \alpha} \rho \pi R^2 v^3 C_{p}(\lambda,\beta) - \frac{1}{\alpha} P_{g}(t) \Delta t + P_{g}(t)\\
\end{split}
\label{eq:euler1}
\end{equation}

    2. Tras esto, se pasa a calcular el valor de la potencia en el instante t=0, es decir, con el valor del ángulo en el instante inicial.
    
    3. Una vez se conoce la potencia inicial, se toma un paso de integración y se decide el número de iteraciones que se van a realizar (cuantas más iteraciones se obtendrá un resultado más preciso, pero más lento, ya que se requiere un mayor número de cálculos). En \ref{eq:pasosIntegracion1}, \ref{eq:pasosIntegracion2} y \ref{eq:pasosIntegracion3} se muestra como sería la resolución numérica de nuestra fórmula por el método de Euler.
    
\begin{equation}
    t=0 \quad \quad \quad \quad \quad \quad P_{g}(\Delta t) = P_{g}(0) + \biggr[\frac{1}{2 \alpha} \rho \pi R^2 v^3 C_{p}(\lambda,\beta) - \frac{P_{g}(0)}{\alpha}\biggr]\Delta t
    \label{eq:pasosIntegracion1}
\end{equation}

\begin{equation}
    t=\Delta t \quad \quad \quad \quad P_{g}(\Delta t + \Delta t) = P_{g}(\Delta t) + \biggr[\frac{1}{2 \alpha} \rho \pi R^2 v^3 C_{p}(\lambda,\beta) - \frac{P_{g}(\Delta t)}{\alpha}\biggr]\Delta t
    \label{eq:pasosIntegracion2}
\end{equation}

\begin{equation}
    t=2 \Delta t \quad \quad \quad P_{g}(2\Delta t + \Delta t) = P_{g}(2\Delta t) + \biggr[\frac{1}{2 \alpha} \rho \pi R^2 v^3 C_{p}(\lambda,\beta) - \frac{P_{g}(2\Delta t)}{\alpha}\biggr]\Delta t
    \label{eq:pasosIntegracion3}
\end{equation}
    
Añadir que se ha utilizado un paso de integración (\textit{$\Delta$t}) de 0.5s y un total de 150 iteraciones por cada vez que se cambia el ángulo.

\section{Red neuronal}

\subsection{¿Qué es una red neuronal?}

Una red neuronal\cite{RedNeuronal} es un modelo que simula de manera simplificada la forma en la que el cerebro procesa información.
Para entender su funcionamiento, lo primero es decir que la unidad básica por la que se compone es la neurona. Estas a su vez, se organizan por capas que se dividen de la siguiente manera:
\begin{itemize}
    \item Capa de entrada: donde se especifica el número de variables de entrada.
    \item Una o más capas ocultas.
    \item Capa de salida: dónde se especifican las variables de salida.
\end{itemize}

Cada una de las neuronas, lo que hace es reproducir una función en base a las entradas recibidas y esta función, a su vez, emite una señal de salida que se envía a la siguiente neurona.

De esta forma, el método de aprendizaje de una red neuronal se resume en ajustar el valor de la función de cada neurona para conseguir la salida deseada a partir de las entradas recibidas.

Por ello concluimos en que para calcular la salida de una neurona, lo primero es calcular es el sumatorio de las entradas multiplicado por el peso sináptico de cada una. Tras esto, se aplica una función de activación\cite{FuncActivacion} sobre el resultado que se encarga de calcular la salida en función de los pesos y las entradas.

Aunque existen varios tipos de funciones de activación, en este proyecto solo vamos a introducir las funciones \textit{reLU}\ref{fig:reLU} y \textit{lineal}\ref{fig:linear}.

\imagen{reLUFunc.png}{Función ReLU.}{0.8}
\label{fig:reLU}

\imagen{linearFunc.png}{Función Lineal.}{1}
\label{fig:linear}

\subsection{Aplicación de las redes neuronales al entrenamiento por refuerzo}

Este proyecto utiliza la arquitectura DQN, el cuál es uno de los algoritmos del aprendizaje por refuerzo. Este algoritmo se basa en utilizar una red neuronal que se encargue de aproximar el valor resultante de cada acción disponible en un estado.

Utilizando el entrenamiento por refuerzo junto con una red neuronal, conseguimos obtener estimaciones más rápidas y mejores, ya que el aprendizaje obtenido para un estado se transfiere a estados similares.

En la imagen \ref{fig:dqn}, se puede ver el funcionamiento de esta arquitectura.

\imagen{dqn.png}{Estructura de una red neuronal.}{1}
\label{fig:dqn}

Para este proyecto, se ha decidido crear una red de cuatro capas:
\begin{itemize}
    \item Una primera capa de 64 neuronas con tipo de función de activación \textit{relu} a la que le pasamos las dimensiones de la variable estado del entorno.
    \item Una segunda capa oculta de 32 neuronas con la misma función de activación.
    \item Una tercera capa idéntica a la anterior.
    \item Por último, se ha añadido una capa de salida a la que se le indica el número de acciones como parámetro de salida y función de activación \textit{linear}, ya que lo que se busca es obtener el estado en sus 3 distintas variantes, según la acción escogida, sin ningún tipo de alteración.
\end{itemize}


\section{Interfaz Web}

Para una mayor facilidad de uso del modelo, orientada sobre todo a usuarios con escasos conocimientos en programación, se propuso como objetivo la realización de una interfaz web sencilla, que permitiese introducir una cantidad de potencia (limitada entre 3 kW y 2925 kW) a la cual se quiera ejecutar el modelo.
Como se puede observar en la imagen \ref{fig:interfaz_web}, la interfaz dispone de un cuadro de texto y dos botones, uno para ejecutar la aplicación pre-entrenamiento y otro para ejecutarla post-entrenamiento.

\imagen{img_capturas_interfaz/Interfaz.png}{Interfaz de la aplicación.}{0.7}
\label{fig:interfaz_web}
\capitulo{4}{Técnicas y herramientas}

Esta parte de la memoria tiene como objetivo presentar las técnicas metodológicas y las herramientas de desarrollo que se han utilizado para llevar a cabo el proyecto.


\section{Metodologías usadas}

\subsection{Metodología Scrum}

La metodología Scrum\cite{MetScrum} es una forma de organización para facilitar la colaboración entre los distintos miembros de un equipo a la hora de realizar un proyecto. Esta metodología consiste en reuniones, roles y herramientas que ayuden al equipo a tener un trabajo más fluido. 

Está dividida en \textit{sprints}, que se definen como las partes en las que se divide el proyecto. En este caso, durante el primer mes se han realizado \textit{sprints} cada una/dos semanas, donde se comentaban los objetivos del proyecto, las fuentes de donde obtener la información y lo que se iba avanzando entre cada una de estas reuniones. Para los siguientes meses, se ha disminuido el número de \textit{sprints} ya que estos eran meramente informativos y para consultar dudas. Esto ha cambiado a lo largo del último mes, donde se ha vuelto a la continuidad de un \textit{sprint} por semana para ultimar los últimos detalles.

\subsection{Técnica Pomodoro}

La técnica Pomodoro\cite{TecPomodoro} es una estrategia que ayuda a optimizar la concentración a la hora de realizar un proyecto, estudiar, etc.

Consiste en dividir el trabajo en espacios de tiempo de unos 25 minutos, realizando una pausa de 5 minutos entre cada uno de ellos. Al mismo tiempo, cada 4 espacios de tiempo se realiza una pausa más larga, de entre 20 - 30 minutos.


\section{Repositorio}

\subsection{GitHub}

GitHub\cite{GitHub} es una plataforma de almacenamiento en la nube para el control de versiones y la colaboración, la cual permite el trabajo con otras personas sobre el mismo proyecto desde cualquier parte del mundo. En ella los desarrolladores pueden almacenar, testear y colaborar al mismo tiempo en el proyecto.

Otra de las opciones que se tuvo en cuenta, fue el uso de GitLab, pero fue descartado debido a que GitHub ya se había utilizado anteriormente y había más familiaridad con su entorno.

El enlace al repositorio es el siguiente:
\href{https://github.com/Di3go99/Control-de-un-molino-eolico-usando-Aprendizaje-por-refuerzo-profundo}{https://github.com/Di3go99/Control-de-un-molino-eolico-usando-Aprendizaje-por-refuerzo-profundo}


\section{Control de versiones}

\subsection{Git}

Git\cite{Git} es hoy en día el sistema de control de versiones más utilizado del mundo. Gracias a Git, podemos llevar un registro de los cambios que realicemos en los archivos de nuestro ordenador y coordinar el trabajo entre varias personas en un proyecto conjunto.

Como ya se ha comentado, el repositorio usado es GitHub que, a través de Git, nos ofrece un control de versiones para nuestro proyecto. 

También se valoró la opción de usar en su lugar la herramienta Subversion, aunque se descartó debido a la facilidad de uso de Git. 


\section{Gestión del proyecto}

\subsection{Microsoft Sharepoint}

Microsoft Sharepoint\cite{Sharepoint} es una web basada en la colaboración que utiliza herramientas de flujo de trabajo para fortalecer el trabajo en equipo en entornos empresariales.Gracias a esta herramienta podemos:

\begin{itemize}
    \item Construir webs privadas, espacios de almacenamiento de archivos y listas.
    \item Personalizar el contenido de nuestras webs.
    \item Compartir avisos importantes, actualizaciones y noticias con gente de nuestro proyecto.
    \item Organizar tu trabajo mediante flujos de trabajo, formularios y listas.
    \item Almacenar y sincronizar tus archivos en la nube para mantenerlos a salvo.
\end{itemize}

Durante la realización del proyecto, se ha utilizado Microsoft Sharepoint tanto para compartir libros, artículos y otros TFG orientativos; como para subir correcciones de la memoria y los anexos del propio proyecto.

Se barajó también la posibilidad de usar ZenHub, ya que permitía una organización por tareas, pero se descartó ya que se quiso simplificar dicha organización y no hacerla tan compleja.


\section{Entorno de desarrollo}

\subsection{Jupyter Notebook}

Jupyter Notebook\cite{Jupyter} es una aplicación en la web a modo de repositorio que permite tanto la creación como la compartición de documentos de código. Es compatible con varios lenguajes y principalmente se usa para el análisis y visualización de datos, además de computación interactiva.

Es una herramienta muy sencilla de usar ya que únicamente dispone de un explorador de archivos y el editor de texto.
Es muy útil a la hora de realizar prototipado de código ya que el código se divide en celdas independientes, lo que permite ejecutar partes del código por separado.

Esta herramienta ha sido la principal en el desarrollo del proyecto ya que se ha utilizado para la completa construcción del código del modelo.


\section{Lenguajes de desarrollo}

\subsection{Python}

Python\cite{Python} es un lenguaje multiplataforma y de código abierto el cual tiene una sintaxis muy simple y sencilla de usar, lo que lo convierte en uno de los lenguajes de programación más utilizados hoy en día.

Entre los campos que Python nos permite trabajar encontramos inteligencia artificial, \textit{machine learning}, \textit{big data} y \textit{data science}, entre otros.

El proyecto completo se ha desarrollado en Python, exceptuando la interfaz web.

\subsection{HTML}

HTML\cite{HTML} es otro lenguaje muy famoso, que se utiliza para el desarrollo web. Este define tanto la estructura como el significado del contenido de una página.

En este proyecto se ha usado esta herramienta junto a Flask para diseñar la interfaz web mediante la cual el usuario interactúa con el modelo del molino.


\section{Librerías y paquetes}

\subsection{Aprendizaje por refuerzo}

\subsubsection{Open AI Gym}

Open AI gym es una librería de Python la cuál dispone de varias funciones para facilitar el aprendizaje por refuerzo. Además, dispone de varios entornos de prueba desarrollados usando distintos algoritmos de entrenamiento y fáciles de configurar.

Esta librería está pensada para aumentar la reproducción de estos algoritmos y proveer al usuario de herramientas que ayuden a entender mejor las bases de la inteligencia artificial.

\subsubsection{Keras RL}

Keras-rl es un paquete de Python que facilita la integración de algoritmos de aprendizaje por refuerzo.

Además, incluye varios ejemplos predefinidos de algoritmos de RL para que el usuario pueda interactuar con ellos y aprender desde una visión más práctica el funcionamiento del aprendizaje por refuerzo.

\subsubsection{Tensorflow}

Tensorflow es una librería de uso libre dedicada al aprendizaje por refuerzo que permite construir modelos de forma más rápida y fácil.

Estas 3 librerías se han utilizado para implementar la funcionalidad necesaria en cuanto al aprendizaje por refuerzo y las redes neuronales.

\subsection{Interfaz-web}

\subsubsection{Flask}

Flask\cite{Flask} es un \textit{framework} de Python que nos permite diseñar aplicaciones web de manera muy sencilla.

Mediante esta herramienta y junto a las plantillas HTML, se ha diseñado la web del modelo.


\section{Documentación}

\subsection{Latex}

\LaTeX es una herramienta de composición de textos, con el objetivo de crear documentos con una gran calidad tipográfica. Suele estar enfocado a la creación de artículos o libros científicos que requieren del uso de expresiones matemáticas.

Para la creación de la memoria del proyecto, se ha hecho uso de este lenguaje mediante la herramienta web Overleaf, la cual esta destinada al uso de Latex, además de ser sencilla de usar y muy intuitiva.

En un principio, se barajó el uso de Microsoft Office Word para esta tarea, de hecho, se comenzó a realizar la memoria en esta herramienta, pero se decidió darle una oportunidad a Latex ya que permite que el documento quede más limpio y ordenado, además de que la UBU dispone de una plantilla ya creada disponible para todos los alumnos.
Por otro lado, también se realizó el cambio por el interés y la curiosidad del alumno por aprender a usar la herramienta.


\section{Comunicación}

La comunicación con los tutores ha sido a través de:

\begin{itemize}
    \item Reuniones a través de Microsoft Teams.
    \item Mensajes de correo electrónico a través de Outlook utilizando el correo de la universidad.
    \item Compartición de documentos y archivos mediante Microsoft Sharepoint.
\end{itemize}


\section{Pruebas}

\subsection{Codacy}

Codacy\cite{Codacy} es una herramienta web que realiza análisis de código para desarrolladores y determina un porcentaje de calidad del código basándose en la complejidad, estilo y funcionamiento del mismo.

Aunque en ningún momento se hizo una subida al repositorio de GitHub con errores en el código ya que se fueron solucionando a través de la propia herramienta Jupyter Notebook, se decidió por realizar un análisis del código a través de Codacy para comprobar duplicidades de código y otras \textit{best practices}.

En la figura \ref{fig:codacy1}, se puede ver un resumen de la calidad del código según Codacy.

\imagen{Codacy.png}{Calidad del código según Codacy.}{1}
\label{fig:codacy1}
\capitulo{5}{Aspectos relevantes del desarrollo del proyecto}

Este apartado pretende recoger los aspectos más interesantes del desarrollo del proyecto, los problemas que han ido surgiendo y de qué forma se ha procedido para solventarlos.


\section{Idea y primeros pasos del proyecto}

La elección de este proyecto se ha visto condicionada, tanto por el interés del alumno en afianzar sus conocimientos en inteligencia artificial, siendo un proyecto muy completo en el que se hace uso de varios elementos de esta rama de la informática; como por la preocupación por la escasa importancia que se le ha dado al uso de las energías renovables en este país y la necesidad de invertir más en ello.

Lo primero que se planteó en las primeras reuniones con los tutores, fueron las herramientas que se iban a utilizar para desarrollar el mismo. Como lenguajes se barajaron dos opciones, Python y Java, aunque se optó rápidamente por la primera opción debido a la sencillez del lenguaje y a que iba a ser una aplicación ejecutada en la web. 
Además, como entorno de desarrollo se eligió Jupyter Notebook, ya que se estimó que iba a ser un proyecto en el cuál se iban a tener que realizar un gran número de pruebas y esta herramienta cumplía esa función perfectamente.

Tras esto se creó tanto un repositorio en GitHub donde subir el código del modelo e ir observando su evolución, como un espacio en Microsoft Sharepoint para compartir libros, artículos, enlaces y TFGs de referencia.

Por último, se estableció la frecuencia con la que se irían realizando las reuniones para tratar los avances realizados en el proyecto.


\section{Formación}

La completa realización del proyecto ha conllevado un continuo trabajo de investigación con el objetivo de resolver las dudas que han ido surgiendo durante la evolución de este.

La mayoría de las herramientas han requerido de una formación previa para su correcta utilización. Desde herramientas como Github, ya utilizada y de la cual se han ampliado conocimientos; como puede ser \LaTeX, donde se partía de cero.

Si hay que destacar algo en lo que se hizo bastante hincapié en cuanto a la formación, fue sobre el qué es, cómo funciona y para qué sirve el aprendizaje por refuerzo.
Sobre este tema se recomendó gran variedad de contenido por parte de los tutores para afianzar los conocimientos del alumno como libros, blogs, vídeos, ejemplos de código, etc. Incluso se tomó una de las reuniones como seminario para explicar de forma detallada y con ejemplos el funcionamiento del aprendizaje por refuerzo.


\section{Desarrollo del modelo}

\subsection{Entorno}

La primera parte que se optó por desarrollar, como en cualquier tipo de modelo de aprendizaje por refuerzo, fue el entorno.

\subsubsection{Primera versión}

En un comienzo, debido al escaso conocimiento inicial del alumno, se optó por realizar un entorno muy básico de 3 acciones discretas en el que se buscaba mantener una variable dentro de un rango de valores.

Esto se hizo para primero comprobar el correcto comportamiento del agente entrenado sobre el entorno antes de entrar en la parte matemática del proyecto con el método de Euler y las ecuaciones de la turbina eólica.

\subsubsection{Segunda versión}

Una vez comprobado que el agente se entrenaba correctamente, se procedió a aumentar el número de acciones para hacer el algoritmo más rápido y preciso.

También se añadió la parte de la integración numérica de la ecuación de la potencia generada por la turbina mediante el método de Euler y se ajustaron algunos parámetros para optimizar el funcionamiento del algoritmo.

\subsection{Modelo DQN}

\subsubsection{Primera versión}

Para la construcción del modelo, se optó por la utilización de una estructura DQN de tipo secuencial, con dos capas ocultas de neuronas que utilizan la función de activación \textit{relu} y una última capa de salida con una función \textit{linear}.

La función de activación se define como el mecanismo mediante el cual las neuronas mandan información a través de la red. La función \textit{relu} es la más usada en los sistemas de redes neuronales ya que hace el modelo más fácil de entrenar y permite conseguir mejores resultados.
Lo que hace esta función es devolver el propio valor que le llega como entrada en caso de ser positivo y cero en caso contrario, lo que provoca que menos neuronas de cada capa se activen y la red tenga un mejor rendimiento.
La última capa usamos una función \textit{linear} ya que lo que buscamos es aproximar un valor de estado real para cada acción. Por ello usamos esta función, la cual nos devuelve el valor de la entrada sin ningún tipo de cambio.

\subsubsection{Segunda versión}

Esta parte del código no se cambió mucho con respecto a la primera versión, quitando algún ajuste en la distribución y el número de neuronas de las capas.

\subsection{Agente DQN}

\subsubsection{Primera versión}

Para el agente se asignó una memoria secuencial y la política de Boltzmann\cite{wiki:Boltzmann}.

Esta política esta diseñada para \textit{action spaces} de tipo discreto. Presupone que todas las acciones tienen un valor asignado y utiiza una función \textit{Softmax} para transformar estos valores en un vector de probabilidades. Tras esto utiliza el vector calculado para realizar pruebas con las posibles acciones.

\subsubsection{Segunda versión}

Para la versión final, se ajustaron algunos parámetros como el límite de pasos guardados en memoria el cual, tras varias pruebas, se estableció en 50000 y el número de pasos de "calentamiento" que se decidió fijar en 1000.

\subsection{Entrenamiento}

Para el entrenamiento del agente, se ha usado la función \textit{fit} de \textit{Tensorflow}. 
Lo primero que hacemos es compilar nuestro agente utilizando el optimizador Adam, que es el de uso común en el aprendizaje por refuerzo, y seguidamente llamamos a la función \textit{fit} en la que, tras varias pruebas, se ha llegado a la conclusión de que 10000 pasos son los suficientes para lograr un buen entrenamiento.

Los optimizadores en las redes neuronales, ayudan a reducir las pérdidas alterando algunos parámetros como el \textit{learning rate}.

\subsection{Pruebas}

Durante todo el desarrollo del modelo se han realizado pruebas para comprobar la funcionalidad de todo lo que se iba implementando.

En cuanto a las pruebas de carácter funcional se han dividido en dos tipos.

\subsubsection{Pruebas pre-entrenamiento}

Para poder ver claramente la diferencia entre el modelo desentrenado y el entrenado, se realizaron una serie de pruebas tomando acciones aleatorias, donde se ve como el modelo no alcanza nunca la potencia deseada.

\imagen{img_pruebas_graficas/Pre_entrenamiento.png}{Gráfica pruebas pre-entrenamiento.}{0.7}
\label{fig:pre}

Como podemos observar en la figura \ref{fig:pre}, los resultados del modelo sin entrenamiento muestran una gráfica en la que nunca se alcanza la potencia de referencia ya que las acciones tomadas son totalmente aleatorias.

\subsubsection{Pruebas post-entrenamiento}

Para las pruebas post entrenamiento, se utilizó el mismo entorno de pruebas que en el caso anterior, pero esta vez dejando que el agente entrenado sea el que tome la decisión sobre qué acción tomar en cada momento, pasándole como referencia un estado inicial aleatorio.

En este caso, para ver el correcto funcionamiento, se realizaron tres tipos de pruebas:

\begin{itemize}
    \item Pruebas sin cambios ni perturbaciones.
    \item Pruebas añadiendo perturbaciones en el viento.
    \item Pruebas añadiendo cambios en la potencia de referencia.
\end{itemize}

\textcolor{red}{Añadir imágenes de las pruebas realizadas tras el entrenamiento.}

\subsection{Interfaz}

Una vez realizadas las pruebas y comprobado que el modelo realizaba bien las predicciones tras el entrenamiento, se precedió a construir la interfaz web.
En un comienzo se empezó a desarrollar utilizando la librería de Python \textit{Tkinter} ya que se pedía una interfaz muy sencilla.

El problema vino a la hora de querer implementar esta interfaz en una aplicación web ya que requería de un conocimiento sobre la herramienta más profundo. Debido a esto se decidió investigar otras formas de desarrollar la interfaz y se optó por el uso del paquete \textit{Flask} el cual, aunque fuese algo más complejo debido a la necesidad de conocimiento de HTML, seguía siendo muy sencillo y fácil de implementar en la web.

Se decidió crear una interfaz sencilla con una plantilla desarrollada en HTML la cual se divide en dos botones y un cuadro de texto. Este cuadro recoge la potencia de referencia y cada uno los botones muestra una gráfica con el resultado pre y post entrenamiento respectivamente.

El desarrollo en python no tiene mucho misterio ya que simplemente se utilizó el mismo código que se usó para realizar las pruebas combinado con las utilidades de \textit{Flask}.
\capitulo{6}{Trabajos relacionados}

La inteligencia artificial es algo que cada vez se esta empezando a implementar más en distintos ámbitos. Tras realizar una búsqueda sobre los distintos proyectos que hay en marcha o que han sido realizados mediante el entrenamiento por refuerzo, encontramos desde juegos de Arcade sencillos hasta investigaciones a nivel global.
A continuación, se va a hablar más detalladamente de alguno de estos ejemplos.


\section{Atari Breakout}

Uno de los ejemplos más básicos en el cual se hace uso del aprendizaje por refuerzo es el videojuego Atari Breakout. En este clásico de las máquinas de Arcade el usuario tiene las opciones de mover una plataforma de izquierda a derecha para hacer rebotar en ella una bola y destruir una serie de ladrillos, evitando a la vez que la bola salga por la parte inferior de la pantalla. En la imagen \ref{fig:atari} se puede ver un ejemplo del juego.

\imagen{atari.png}{AtariBreakout}{0.3}
\label{fig:atari}

En este caso, lo que se busca es entrenar a un agente para que sea capaz de jugar al juego automáticamente por nosotros. No tiene ninguna utilidad específica, pero es un buen ejemplo para entender el funcionamiento del aprendizaje por refuerzo.


\section{Control de un reactor de fusión nuclear}

Entrando de lleno en la aplicación real del aprendizaje por refuerzo en proyectos de gran escala, podemos encontrar como el \textit{Swiss Plasma Center - EPFL}\cite{SPC} ha conseguido entrenar un algoritmo capaz de controlar el sobrecalentamiento del plasma contenido en un reactor de fusión nuclear.

La red neuronal del modelo entrenado coge, diez mil veces por cada segundo, unas 90 medidas distintas de distintos parámetros que describen la forma y posición del plasma y usa estos datos para ajustar 19 imanes. Estos imanes crean un campo electromagnético que evita que el plasma roce las paredes del reactor y este pueda quedar dañado.


\section{Coches de conducción auto-independiente}

Este seguramente sea uno de los proyectos de R.L. más interesantes a día de hoy. Desde hace varios años se lleva intentado conseguir, por medio de esta herramienta, coches capaces de funcionar sin necesidad de un conductor.

La empresa que encabeza este proyecto es \textbf{Waybe}\cite{Wayve}, aunque hay muchas otras que también están interesadas en desarrollar esta tecnología. Tras 15 años de investigación, esta empresa ha conseguido que un coche entrenado en las calles de Londres, sea capaz de conducirse automáticamente en las ciudades de Cambridge, Coventry, Leeds, Liverpool y Manchester sin necesidad de un entrenamiento adicional. Su objetivo es construir el primer vehículo capaz de conducirse por si solo en 100 ciudades distintas.

En el siguiente vídeo, se puede ver el primer coche al que se consiguió, en tan solo 20 minutos, entrenar para que se mantuviese en carretera por sí solo:
\href{https://youtu.be/eRwTbRtnT1I}{https://youtu.be/eRwTbRtnT1I}.

Desgraciadamente esto les esta llevando más tiempo del esperado por lo que han decidido optar por, en vez de crear pequeñas redes neuronales y luego conectarlas manualmente, construir una red neuronal que sea capaz de convertir una señal de entrada (por ejemplo una cámara que grabe la carretera por delante del vehículo) en una señal de salida (como puede ser cambiar la dirección de las ruedas o frenar).

Esto es conocido como el aprendizaje \textit{end-to-end}\cite{end_to_end}, el cual ha sido usado en múltiples ocasiones para entrenar otras inteligencias artificiales como es el caso de \textit{AlphaZero}\cite{AlphaZero}.

\capitulo{7}{Conclusiones y Líneas de trabajo futuras}

A continuación, se exponen una serie de conclusiones sobre el proyecto realizado y las líneas futuras por las que podría dar continuidad al proyecto.


\section{Conclusiones}

Tras la realización de este proyecto, me llevo una sensación positiva tanto por todo lo que he aprendido sobre las diferentes herramientas y puntos temáticos que utilizado e investigado, como por la implicación en la mejora de algo como las energías renovables.

Al mismo tiempo, desde el comienzo hasta el final de la construcción del proyecto, me he dando cuenta de la cantidad de cosas aprendidas en la carrera que he aplicado, incluso inconscientemente, desde el uso de un repositorio hasta el modelado de diagramas \textit{UML}.
Donde más he tenido esta sensación de facilidad, ha sido a la hora de trabajar con \textit{Python} que, a pesar de ser un lenguaje del cual no poseía muchos conocimientos, el hecho de haber trabajado durante la carrera con distintos lenguajes de programación (unos más parecidos y otros menos) me ha ayudado considerablemente a su comprensión y su uso.

Como conclusión, diría que ha sido un proyecto en el que el hecho de haber abordado un tema en el que se disponían de los conocimientos justos, como es el aprendizaje por refuero, me ha servido considerablemente para comprender mucho mejor el funcionamiento de la inteligencia artificial, a la par que ha despertado un interés en mi tanto por seguir explorando este campo, cómo por descubrir otros nuevos.


\section{Líneas futuras}

Aunque es cierto que el proyecto es un modelo muy sencillo al cuál aún le quedan años de investigación y desarrollo, puede servir como idea para impulsar otros proyectos relacionados con la mejora de las energías renovables a través de la inteligencia artificial.

A continuación se describen una serie de mejoras que podrían aplicarse al modelo.

\subsection{Mejora de la matemática}

Para la parte matemática, se decidió por simplificar el modelo a una ecuación más sencilla ya que, el tener en cuenta todos los parámetros y variables que influyen en el comportamiento de una turbina eólica real, complicaba mucho el proyecto.

Una mejora considerable sería aplicar toda esta parte matemática omitida y crear así un modelo más preciso y, al mismo tiempo, que se corresponda con la realidad.

\subsection{Simulador de viento real}

Una mejora que se pensó pero que, finalmente, no se pudo llevar a cabo fue la modelización de un simulador de viento.
Se pensó esta idea para comprobar la respuesta del molino ante la entrada de perturbaciones que pudiesen variar el ángulo en un entorno real y así recopilar una serie de pruebas más fidedignas.

\subsection{Más opciones de control}

Cómo se ha explicado anteriormente, el único parámetro con el cual el usuario puede interactuar es la potencia. Una buena mejora sería dotar al usuario de más opciones para poder realizar otro tipo de pruebas y tener opción a observar mayor número de resultados distintos.

Un ejemplo podría ser el poder controlar la velocidad de giro de las aspas, pudiendo así observar los resultados obtenidos ante velocidades variables.

\subsection{Mejora de la interfaz}

Una última mejora, quizás menos importante, sería rediseñar la parte de la interfaz de usuario.
Una idea es que se podría cambiar el diseño, incluso implementando un modelo visual animado del molino que interactuase de una manera u otra según los parámetros introducidos por el usuario.


\bibliographystyle{plain}
\bibliography{bibliografia}

\end{document}
