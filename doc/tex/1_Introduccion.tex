\capitulo{1}{Introducción}

El interés por el uso y el funcionamiento de las energías renovables ha crecido mucho en los últimos años y, en particular, la energía producida por el viento.
La mejora en la eficiencia de las plantas de producción eólica sigue siendo un desafío pero también una necesidad absoluta.

Al mismo tiempo, se han realizado un gran número de avances en cuanto a inteligencia artificial\cite{wiki:IA}, la cual vemos cada vez más implementada en distintos aparatos que usamos en la vida cotidiana. Estos avances han llevado a la aparición del aprendizaje por refuerzo profundo\cite{RL}, el cual consiste en el aprendizaje por parte de un agente computacional a tomar decisiones por un método de prueba y error, pretendiendo alcanzar una solución óptima a un problema.

La posibilidad de combinar esto con las ecuaciones de una turbina eólica, sumado al problema de los costes que conlleva el tener un modelo físico de la propia turbina para realizar pruebas, han llevado a la idea de este proyecto, en el cual se modeliza el funcionamiento de un molino eólico a través de distintas herramientas de software.


\section{Estructura de la memoria}

\begin{itemize}
    \item \textbf{Introducción}: breve descripción del proyecto y del problema a resolver, junto a la estructura de la memoria y los materiales adjuntos.
    \item \textbf{Objetivos del proyecto}: objetivos generales, técnicos y personales que se pretenden lograr con la realización del proyecto.
    \item \textbf{Conceptos teóricos}: Breve descripción de una serie de conceptos que son necesarios conocer para la correcta comprensión del proyecto.
    \item \textbf{Técnicas y herramientas}: aquí se describen la serie de elementos que se han utilizado durante todo el proyecto, así como distintas técnicas que han permitido facilitar el desarrollo del mismo.
    \item \textbf{Aspectos relevantes del desarrollo del proyecto}: descripción general de la realización del proyecto.
    \item \textbf{Trabajos relacionados}: trabajos parecidos o similares al desarrollado.
    \item \textbf{Conclusiones y líneas de trabajo futuras}: pensamientos y opinión personal tras la realización del proyecto, así como mejoras y posibilidades de seguir desarrollando y mejorando el proyecto por otras ramas.
\end{itemize}

\section{Materiales adjuntos}

\begin{itemize}
    \item Link al repositorio de GitHub con el código: \href{https://github.com/Di3go99/GII-21.35-Control-de-un-molino-eolico-usando-Aprendizaje-por-refuerzo-profundo}{https://github.com/Di3go99/GII-21.35-Control-de-un-molino-eolico-usando-Aprendizaje-por-refuerzo-profundo}.
    \item Memoria detallada del proyecto.
    \item Anexos.
\end{itemize}