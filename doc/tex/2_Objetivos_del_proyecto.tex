\capitulo{2}{Objetivos del proyecto}

En este apartado se indican los distintos objetivos que han motivado al alumno a la realización del proyecto.

\section{Objetivos generales}

\begin{itemize}
    \item Modelar el comportamiento de una turbina eólica a partir de un algoritmo de aprendizaje por refuerzo, el cuál sea capaz de ajustarse automáticamente a una demanda de energía dada.
    \item Calcular y aplicar las ecuaciones que intervienen en el funcionamiento de una turbina eólica.
    \item Diseñar una interfaz-web sencilla para interactuar con el software del modelo.
    \item Facilitar el entendimiento de la respuesta del molino mediante gráficas.
\end{itemize}


\section{Objetivos técnicos}

\begin{itemize}
    \item Desarrollar una aplicación el Python\cite{Python} mediante la plataforma Jupyter Notebook\cite{Jupyter}.
    \item Utilizar los paquetes de \textit{Keras-rl} y \textit{Tensorflow} para el aprendizaje por refuerzo.
    \item Utilizar los paquetes de \textit{Numpy} y \textit{MatbPlot} para la representación gráfica de resultados matemáticos.
    \item Utilizar Flask\cite{Flask} y HTML\cite{HTML} para el desarrollo de una interfaz web sencilla.
    \item Utilizar una estructura de red neuronal DQN\cite{wiki:DQN} para entrenar un agente.
    \item Utilizar GitHub\cite{GitHub} como repositorio para ir publicando las distintas versiones del proyecto y tener un registro de su evolución.
    \item Aplicar la metodología Scrum\cite{MetScrum} para organizar las reuniones y dividir las tareas en \textit{Sprints}.
    \item Utilizar la aplicación web Microsoft Sharepoint\cite{Sharepoint} para compartir artículos, foros, libros y cualquier tipo de documentación que pueda ser de ayuda para la realización del proyecto.
    \item Documentar toda la información del proyecto utilizando \LaTeX\cite{wiki:latex}.
\end{itemize}


\section{Objetivos personales}

\begin{itemize}
    \item Comprender el funcionamiento del aprendizaje por refuerzo, así como entender la importancia del mismo y saber aplicarlo a un modelo software.
    \item Profundizar en el campo de la inteligencia artificial.
    \item Aplicar el mayor número de conocimientos obtenidos durante la carrera.
    \item Aprender a utilizar todas las aplicaciones mencionadas en las \textit{Técnicas y herramientas} y afianzar conocimientos en aquellas usadas previamente.
\end{itemize}