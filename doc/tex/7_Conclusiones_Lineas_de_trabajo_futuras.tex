\capitulo{7}{Conclusiones y Líneas de trabajo futuras}

A continuación, se exponen una serie de conclusiones sobre el proyecto realizado y las líneas futuras por las que podría dar continuidad al proyecto.


\section{Conclusiones}

Tras la realización de este proyecto, me llevo una sensación positiva tanto por todo lo que he aprendido sobre las diferentes herramientas y puntos temáticos que utilizado e investigado, como por la implicación en la mejora de algo como las energías renovables.

Al mismo tiempo, desde el comienzo hasta el final de la construcción del proyecto, me he dando cuenta de la cantidad de cosas aprendidas en la carrera que he aplicado, incluso inconscientemente, desde el uso de un repositorio hasta el modelado de diagramas \textit{UML}.
Donde más he tenido esta sensación de facilidad, ha sido a la hora de trabajar con \textit{Python} que, a pesar de ser un lenguaje del cual no poseía muchos conocimientos, el hecho de haber trabajado durante la carrera con distintos lenguajes de programación (unos más parecidos y otros menos) me ha ayudado considerablemente a su comprensión y su uso.

Como conclusión, diría que ha sido un proyecto en el que el hecho de haber abordado un tema en el que se disponían de los conocimientos justos, como es el aprendizaje por refuero, me ha servido considerablemente para comprender mucho mejor el funcionamiento de la inteligencia artificial, a la par que ha despertado un interés en mi tanto por seguir explorando este campo, cómo por descubrir otros nuevos.


\section{Líneas futuras}

Aunque es cierto que el proyecto es un modelo muy sencillo al cuál aún le quedan años de investigación y desarrollo, puede servir como idea para impulsar otros proyectos relacionados con la mejora de las energías renovables a través de la inteligencia artificial.

A continuación se describen una serie de mejoras que podrían aplicarse al modelo.

\subsection{Mejora de la matemática}

Para la parte matemática, se decidió por simplificar el modelo a una ecuación más sencilla ya que, el tener en cuenta todos los parámetros y variables que influyen en el comportamiento de una turbina eólica real, complicaba mucho el proyecto.

Una mejora considerable sería aplicar toda esta parte matemática omitida y crear así un modelo más preciso y, al mismo tiempo, que se corresponda con la realidad.

\subsection{Simulador de viento real}

Una mejora que se pensó pero que, finalmente, no se pudo llevar a cabo fue la modelización de un simulador de viento.
Se pensó esta idea para comprobar la respuesta del molino ante la entrada de perturbaciones que pudiesen variar el ángulo en un entorno real y así recopilar una serie de pruebas más fidedignas.

\subsection{Más opciones de control}

Cómo se ha explicado anteriormente, el único parámetro con el cual el usuario puede interactuar es la potencia. Una buena mejora sería dotar al usuario de más opciones para poder realizar otro tipo de pruebas y tener opción a observar mayor número de resultados distintos.

Un ejemplo podría ser el poder controlar la velocidad de giro de las aspas, pudiendo así observar los resultados obtenidos ante velocidades variables.

\subsection{Mejora de la interfaz}

Una última mejora, quizás menos importante, sería rediseñar la parte de la interfaz de usuario.
Una idea es que se podría cambiar el diseño, incluso implementando un modelo visual animado del molino que interactuase de una manera u otra según los parámetros introducidos por el usuario.