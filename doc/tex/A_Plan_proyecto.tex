\apendice{Plan de Proyecto Software}

\section{Introducción}

En este apéndice, se van a explicar y comentar todos los aspectos relacionados con la planificación del proyecto desde el punto de vista del desarrollo software.

La fase de planificación se divide en:
\begin{itemize}
    \item Planificación temporal.
    \item Estudio de viabilidad.
\end{itemize}

La primera parte esta destinada, mayormente, a la distribución del proyecto en cuanto al tiempo, la división en tareas del mismo y la organización tanto del alumno como de los tutores para su elaboración.
La segunda parte, se centra en el estudio de la rentabilidad del proyecto. Esta, se divide en dos puntos importantes:
\begin{itemize}
    \item Viabilidad económica: en esta sección se hace un baremo entre los costes y los beneficios del proyecto.
    \item Viabilidad legal: está mayormente orientada a temas de licencias y protección de datos.
\end{itemize}


\section{Planificación temporal}

En cuanto a la organización del proyecto, al comienzo del mismo se planteó utilizar la metodología \textit{Scrum} para organizar las tareas y la regularidad de las reuniones. 
Se ha procurado seguir las pautas establecidas sobre la organización de los \textit{sprints} en todo momento, aunque también es cierto que, debido a los problemas que han ido surgiendo, ha sido imposible seguir esto a rajatabla.

\begin{itemize}
    \item Estrategia de desarrollo incremental a través de \textit{sprints}.
    \item Sprints cada 1-2 semanas según la etapa del proyecto.
    \item En cada \textit{sprints}, se realizaba una entrega de una parte del proyecto funcional.
    \item Se han realizado reuniones en cada uno de los \textit{sprints}.
    \item En cada uno de los \textit{sprints} se han generado una serie de tareas a realizar.
\end{itemize}

La estimación se realizó mediante una serie de \textit{story points}, cuya traducción temporal aparece en la tabla \ref{tabla:storypoint}

\tablaSmall{Story Point Cheat Sheet}{l c}{storypoint}
{\multicolumn{1}{l}{Story points} & Estimación temporal\\}{ 
1 & 30 minutos\\
3 & 2 horas\\
5 & 3 horas\\
8 & 5 horas\\
13 & 10 horas\\
20 & 20 horas\\
40 & 32 horas\\
} 

Los \textit{sprints} realizados han sido los siguientes:

\subsection{Sprint 0 (15/02/2022 - 28/02/2022)}
\begin{itemize}
    \item Elección de la temática del proyecto entre las ofertadas por la UBU.
    \item Primera reunión con los tutores, para comentar las bases generales del TFG, plazos de entrega, división del proyecto en \textit{sprints}.
    \item Firma, por parte de los tutores, del documento que acreditaba que el alumno había sido asignado a este proyecto.
\end{itemize}

\subsection{Sprint 1 (01/03/2022 - 15/03/2022)}
\begin{itemize}
    \item Investigación exhaustiva sobre el tema del proyecto.
    \item Lectura de libros y artículos recomendados por los tutores para reforzar conocimientos.
    \item Pequeño seminario en el cuál se aclararon dudas y se explicaron de forma más concreta ciertos puntos más difíciles de comprender.
    \item Creación del repositorio de GitHub.
    \item Creación del Sharepoint.
    \item Descarga e instalación de las herramientas y el entorno de desarrollo.
    \item Establecimiento de los objetivos y metas del proyecto.
\end{itemize}

\subsection{Sprint 2 (15/03/2022 - 31/03/2022)}
\begin{itemize}
    \item instalación de paquetes y librerías necesarias y comienzo de la parte de programación del proyecto.
    \item Investigación sobre el uso del editor de textos \LaTeX y primer contacto con la herramienta.
    \item Lectura e investigación sobre distintos Trabajos de Fin de Grado disponibles.
    \item Comienzo de la memoria con los Conceptos teóricos que se aprendieron hasta el momento y las Técnicas y herramientas que se iban usando (Estos dos apartados de la memoria se han ido completando a lo largo del proyecto según se aprendían conceptos y se usaban nuevas herramientas).
\end{itemize}

\subsection{Sprint 3 (01/04/2022 - 15/04/2022)}
\begin{itemize}
    \item Finalización del desarrollo del entorno de la parte de aprendizaje por refuerzo.
    \item Investigación sobre cómo construir la red neuronal y primeros pasos del desarrollo.
\end{itemize}

\subsection{Sprint 4 (16/04/2022 - 30/04/2022)}
\begin{itemize}
    \item Creación tanto de la red neuronal como del agente a entrenar.
    \item Primeras pruebas de la calidad del entrenamiento sobre el modelo.
    \item Contrastación de las pruebas con los tutores, para recibir soluciones de errores y sugerencias de mejoras.
    \item Continuación de la memoria del proyecto.
\end{itemize}

\subsection{Sprint 5 (01/05/2022 - 15/05/2022)}
\begin{itemize}
    \item Finalización de la funcionalidad de la parte de aprendizaje por refuerzo, a falta de alguna mejora.
    \item Investigación sobre el funcionamiento de una turbina eólica tanto teórica, como matemáticamente.
    \item Contrastación de lo desarrollado hasta el momento con otros ejemplos de entrenamiento por refuerzo disponibles en internet.
    \item Investigación sobre el método de Euler.
\end{itemize}

\subsection{Sprint 6 (16/05/2022 - 31/05/2022)}
\begin{itemize}
    \item Implementación de la matemática del molino en el código.
    \item Realización de varias pruebas para comprobar la funcionalidad del algoritmo.
    \item Implementación de algunas mejoras sugeridas en la reunión Scrum.
    \item Comienzo de desarrollo de los anexos.
    \item Desarrollo de la interfaz web.
    \item Creación de lista con detalles y mejoras a falta de implementar en el proyecto.
\end{itemize}

\subsection{Sprint 7 (01/06/2022 - 15/06/2022)}
\begin{itemize}
    \item Implementación de mejoras restantes.
    \item Resolución de ciertas dudas sobre la memoria y algunos puntos de las bases de la realización del TFG pendientes de aclarar.
    \item Continuación tanto de la memoria, como de los anexos.
\end{itemize}

\subsection{Sprint 8 (15/06/2022 - 7/07/2022)}
\begin{itemize}
    \item Finalización y corrección de la memoria.
    \item Creación del documento para la presentación ante el tribunal.
\end{itemize}


\section{Estudio de viabilidad}

A continuación, se realiza un análisis económico en el hipotético caso de que el proyecto se llegase a utilizar en el mercado laboral. 

Aún así, dejar claro que para ello sería necesario de 1 a 2 años más de desarrollo del proyecto para que fuese viable.

\subsection{Viabilidad económica}

En este apartado de la viabilidad, se analizan tanto los costes como los beneficios que podría aportar el proyecto en el caso de ser usado en un entorno empresarial real.

\subsubsection{Coste de software}

El coste de software se refiere al precio de las licencias de los programas, aplicaciones y suscripciones web que se han utilizado durante el desarrollo del proyecto. Los costes de software de nuestro proyecto son los que aparecen en la tabla \ref{tabla:costesoftware}

\tablaSmall{Coste de software}{l c}{costesoftware}
{ \multicolumn{1}{l}{Concepto} & Coste\\}{ 
Microsoft Windows 10 Pro & 11.50 €\\
Codacy & 14.40 €\\
\textbf{Total} & \textbf{25.90 €}\\
} 

\subsubsection{Coste de personal}

El coste de personal se refiere al salario que habría que pagarle al desarrollador junior por las horas imputadas. En la tabla \ref{tabla:costepersonal} podemos ver una estimación de dichos costes.

\tablaSmall{Coste de personal}{l c}{costepersonal}
{ \multicolumn{1}{l}{Concepto} & Pagos\\}{ 
Salario neto del empleado & 1000.00 €\\
IRPF (19\%) & 360.00 €\\
Seguridad social (28,3\%) & 537.00 €\\
Salario bruto del trabajador & 1897.00 €\\
\textbf{Total} & \textbf{3794.00 €}\\
}

\subsubsection{Coste de material}

El coste de material son los productos de hardware que han sido necesarios para la realización del proyecto. En este caso, como se puede ver en la tabla \ref{tabla:costematerial}, el único gasto en material sería el precio del ordenador utilizado para el desarrollo.

\tablaSmall{Coste de material}{l c}{costematerial}
{ \multicolumn{1}{l}{Concepto} & Coste\\}{ 
Ordenador & 700.00 €\\
\textbf{Total} & \textbf{700.00 €}\\
} 

\subsubsection{Beneficios}

Debido a que este es un simple proyecto de fin de grado destinado al aprendizaje y a poner a prueba al alumno, no se busca obtener ningún beneficio de ello.

\subsection{Viabilidad legal}

A continuación, se pasa a analizar las leyes que podrían afectar al proyecto desarrollado, en caso de utilizarse a nivel laboral.

\subsubsection{Licencias}

Uno de los mayores problemas que suelen surgir con este tipo de proyectos, es el tema de las licencias de software.
En este caso, casi todo el software utilizado es de uso libre y, en el caso de haberse utilizado algún software de pago, todas las licencias han sido pagadas, por lo que no supondría un problema legal en este caso.

\subsubsection{Protección de datos}

Otro tema que también es bastante relevante en cuanto a este apartado, es el tema de la protección de datos personales.
En este proyecto, no se solicita al usuario de ningún dato personal, por lo que tampoco habría que preocuparse en ese aspecto.
