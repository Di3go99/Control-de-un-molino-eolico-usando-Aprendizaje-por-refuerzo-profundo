\apendice{Especificación de diseño}

\section{Introducción}

Mediante la especificación de diseño, se aporta un mayor entendimiento del proyecto al equipo de desarrollo y se definen los datos que usa la aplicación, el diseño procedimental y la arquitectura.

\section{Diseño de datos}

En el caso de este proyecto, no es necesario el manejo de datos.

Como mucho podríamos hablar de la memoria secuencial que utiliza el agente. En esta se van almacenando los resultados de los entrenamientos pasados, hasta un límite que definamos nosotros.
Cuando el backlog de los entrenamientos llega a este límite, la información del último entrenamiento sustituirá a la más antigua en memoria, ya que presupone que esta nueva información es más útil y precisa.

\section{Diseño procedimental}

En este apartado se procede a detallar la secuencia que sigue la ejecución del modelo.

En el diagrama de secuencia de la figura \ref{fig:DiagramaSecuencia}, se aprecian las interacciones entre los objetos que conforman el modelo.

\imagen{DiagramaSecuenciaTFG.png}{Diagrama de secuencia del modelo}{1.0}
\label{fig:DiagramaSecuencia}

\section{Diseño de interfaces}

Para este proyecto, se pedía la creación de una interfaz sencilla que permitiese al usuario introducir una cantidad de potencia.

En la figura \ref{fig:interfaz}, se muestra otro diagrama del planteamiento de la interfaz.

\imagen{Interfaz.png}{Diagrama de la interfaz de la aplicación}{1}
\label{fig:interfaz}