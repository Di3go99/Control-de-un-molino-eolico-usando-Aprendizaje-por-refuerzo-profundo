\apendice{Documentación técnica de programación}

\section{Introducción}

En este apéndice, se procede a detallar la documentación técnica de la programación del modelo.

Aquí se incluyen la instalación del entorno de desarrollo, así como la compilación y ejecución del proyecto, la estructura de directorios, el manual del programador y las pruebas realizadas.

\section{Estructura de directorios}

El repositorio cuenta con la siguiente distribución:
\begin{itemize}
    \item \textbf{/}: es el directorio raíz, donde están almacenados todos los archivos del proyecto junto al fichero \textit{README}.
    \item \textbf{/docs}: aquí podemos encontrar toda la documentación del proyecto (memoria y anexos).
    \item \textbf{/docs/img}: esta carpeta contiene las imágenes utilizadas en la documentación.
    \item \textbf{/docs/tex}: contiene los ficheros de \LaTeX de cada uno de los apartados de la documentación.
    \item \textbf{/src}: aquí encontramos los archivos \textit{.ipynb} con todo el código desarrollado. Dentro podemos encontrar el archivo \textit{Paquetes.ipynb} con los paquetes necesarios para ejecutar la aplicación, el archivo \textit{Pruebas.ipynb} con el entorno de pruebas y el archivo \textit{Turbina.ipynb}, desde el cual ejecutamos la aplicación.
    \item \textbf{/src/templates}: contiene el fichero \textit{.html} de la interfaz web.
\end{itemize}


\section{Manual del programador}

El objetivo del manual del programador es explicar el funcionamiento del proyecto a una persona que no haya participado en su realización y necesite conocer esta información. Este puede ser el caso de alguien que se haya contratado para el mantenimiento del sistema y necesite conocer de su funcionamiento.

\subsection{Entorno de desarrollo}

Para trabajar con el proyecto correctamente, es necesario instalar las siguiente serie de programas y dependencias:

\begin{itemize}
    \item \textbf{\href{https://www.python.org/downloads/}{Python}}: es el lenguaje en el que esta desarrollado casi todo el modelo. Es necesario para obtener todas las librerías para el correcto funcionamiento.
    \item \textbf{\href{https://jupyter.org/install}{Jupyter Notebook}}: es el IDE utilizado para el desarrollo del modelo. Es el recomendado para trabajar con el código, aunque se pueden usar otras aplicaciones.
     \item \textbf{\href{https://git-scm.com/book/en/v2/Getting-Started-Installing-Git}{Git}}: es necesario para hacer uso del repositorio de GitHub y poder obtener la última versión del proyecto.
\end{itemize}

\subsection{Descarga del código fuente}

Para obtener el código fuente y poder trabajar con él se deben seguir los siguientes pasos:

\begin{enumerate}
    \item Abrimos la terminal de Git Bash.
    \item Nos situamos, mediante el comando \textbf{cd}, en el directorio donde queramos guardar el proyecto.
    \item Utilizamos el comando \textbf{git clone} para clonar los documentos del repositorio en nuestro directorio. El comando completo para este caso es: \textbf{git clone https://github.com/Di3go99/Control-de-un-molino-eolico-usando-Aprendizaje-por-refuerzo-profundo.git}
    \item Una vez se complete la descarga, dispondremos de los archivos en nuestro directorio.
\end{enumerate}

Para trabajar con el código del proyecto, simplemente tendremos que abrir Jupyter Notebook y acceder al directorio donde hayamos guardado el proyecto. Tras esto, entramos en la carpeta \textbf{/src} y abrimos el archivo \textit{Turbina.ipynb}, donde encontraremos el código completo de la aplicación.


\section{Compilación, instalación y ejecución del proyecto}
\label{compilacion}

Lo primero que debemos hacer es instalar los paquetes necesarios para el código, los cuales podemos observar en la figura \ref{fig:paquetes}. Para ello, abrimos Jupyter Notebook, navegamos hasta la carpeta \textbf{/src} y abrimos el archivo \textit{Paquetes.ipynb}.

\imagen{Paquetes.png}{Paquetes necesarios para el modelo}{0.4}
\label{fig:paquetes}

Una vez cargado el archivo, pulsamos en el botón superior que dice \textbf{Run} y esperamos a que termine la ejecución de la celda.

Por último, volvemos a la carpeta \textbf{/src} y abrimos el archivo \textit{Turbina.ipynb}. Ahora hacemos clic en \textbf{Cell > Run All} (como puede verse en la figura \ref{fig:ejecucion}) para que se ejecuten todas las celdas de código.

\imagen{RunAll.png}{Botón para ejecutar todas las celdas}{0.8}
\label{fig:ejecucion}


\section{Pruebas del sistema}

Las pruebas realizadas sobre el proyecto, han sido pruebas del tipo ensayo-error realizadas durante todo el proceso de creación del mismo. A lo largo del transcurso del proyecto se han ido tomando los ajustes necesarios según el resultado de dichas pruebas.

En cuanto a las pruebas de calidad del código, se han realizado varios chequeos en Codacy\cite{Codacy} de cada una de las subidas al repositorio a lo largo del último mes. En la figura \ref{fig:codacy2} se puede ver un resumen con la calidad del código según Codacy

\imagen{Codacy.png}{Resumen de la calidad del código según Codacy}{1.1}
\label{fig:codacy2}