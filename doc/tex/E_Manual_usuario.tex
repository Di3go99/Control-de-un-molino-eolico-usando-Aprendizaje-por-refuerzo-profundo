\apendice{Documentación de usuario}

\section{Introducción}

A continuación, se proceden a detallar, tanto los requisitos necesarios para que el usuario pueda usar el modelo, como los pasos para la instalación y el manual de uso del mismo.


\section{Requisitos de usuarios}

El usuario va a requerir del siguiente software.

\begin{itemize}
    \item \textbf{Python}: es el lenguaje en el que esta desarrollado casi todo el modelo. Es necesario para obtener todas las librerías para el correcto funcionamiento.
    \item \textbf{Jupyter Notebook}: es el IDE utilizado para el desarrollo del modelo. Es el recomendado para trabajar con el código, aunque se pueden usar otras aplicaciones.
    \item \textbf{última versión}: es necesario que el usuario disponga de la última versión del proyecto descargada en su ordenador.
\end{itemize}


\section{Instalación}

Para la instalación, simplemente hay que descomprimir el archivo .zip con el código, ejecutarlo y abrir en el navegador el link que podemos encontrar en el documento Link.txt del repositorio.


\section{Manual del usuario}

El modelo dispone de una interfaz muy sencilla, en la que el usuario simplemente debe rellenar un cuadro de texto y pulsar uno de los botones que aparecen en pantalla.

El funcionamiento es el siguiente:
\begin{itemize}
    \item \textbf{Cuadro de texto \textit{Potencia}}: el usuario debe rellenar, antes de pulsar ningún botón, este cuadro de texto con la cantidad de potencia que deseamos que el modelo alcance.
    \item \textbf{Botón \textit{Ejecutar sin entrenar}}: mediante este botón, al usuario se le mostrará una gráfica del comportamiento del modelo previo al entrenamiento del agente.
    \item \textbf{Botón \textit{Entrenar y ejecutar}}: mediante este botón, al usuario se le mostrará una gráfica del comportamiento del modelo tras el correcto entrenamiento del agente.
\end{itemize}

